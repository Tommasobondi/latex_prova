\documentclass{article}
\usepackage[utf8]{inputenc}
\usepackage{hyperref}
\usepackage{natbib}
\usepackage{xcolor}
%esiste funzione per rinominare funzioni, in modo da poterle scrivere più velocemente nel testo
%nb. cosi facendo, io posso poi togliere tutto ciò che è rosso inserendo "%" davati a mia finzione rinominata
%https://www.overleaf.com/learn/latex/Commands
\newcommand{\rosso}{\textcolor{red}}
%riscordfa che con hyperref è possibile materializzare i richiami nel testo, rendendoli interattivi

\title{Formule}
\author{Tommaso Leone Bondi }
\date{May 2021}

\begin{document}

\maketitle
\begin{abstract}
In questo latex inposteremo formule mediante la compilazione automatica
%per mettere in grassetto funz
\textbf{fhdbsnabsvfsnffADS}NADSNJCNJCSAJNCAN
%per mettere in corsivo funz 
\textit{gvdcbhasnxmkcsndbsvdcasn}CSNNJCASNJCS
%ovviamente poi è sembre comodo rinominarla come visto per coloro in modo da semplificarne la gestione poi
\end{abstract}

\section{Introduction}
%inserire equazioni in testo, son singole
%mediante il simbolo $$ inserisco formule in un testo, scrivendola tra i due simboli
section with number, come vedremo inserirò formule all'interno del documento latex, formule quali $\sum$ e altre
\section*{chap 1}
section no numbe
%per le cit, per richiamarle si utilizza cite, vedi il richiamo alla bibliografia
sto inserendo la cit di bus, avendo inserito \citep{brus et al.} sarà davvero veloce compilare bibliografia
\subsection{sub 1}
subsection number
%https://it.overleaf.com/learn/latex/Using_colours_in_LaTeX
cambiamo ora colore del testo con funzione textcolor
%i colori sono bastati sul pacchetto xcolor, testo colorota va binserito tra la seconda parentesi, il colore è nella prima
\textcolor{red}{jndfwbkyuafnSM,LMFNUBEHLQRWIJ} kmadda \rosso{njbhcnadbda}
%nb. sono entrambe colorate con textcolor ma ora comando è rinominato
\subsection*{sub 2}
sunbsec no number \citep{Adorno 1960} ( caso cit sbagliata)jhjdfsa
\section{formula}
%come scrivere formule, funzione ovviamemte si apre con begin
%https://en.wikibooks.org/wiki/LaTeX/Mathematics -> per vedere come si scrivono formule
\begin{equation}
%times è funzione per moltiplicazione, sum per somma, log per logaritmo
%per mettere il pedice nella formula devo usare underscore, associato a lettere a cui va associato  il pedice
H= - \sum  p_i \times  \log{p_{ijm}}
\label{equation}
%si provi ora a scrivere sommatoria con sommatoria da ... a...
%apice si inserisce con ^ associato a lettera/parola a cui si riferisce
\end{equation}
\begin{equation}
  \sum_{i=1}^{k} 
  \label{eqaution 2}
\end{equation}
\section{discussione}
%io scrivo un equazione, che menzionerò successivamente, ad esempio quella in riga 29, come si vede in riga trenta metto lable a equazione
come abbiamo visto nella sezione formula
\ref{equation}

%per richiamare inserisco ref, tra graffa nome che ho assegnato in lable a cio che voglio richiamare
come abbiamo visto in sezione formula, nella formula due \ref{eqaution 2}, si può inserire ref in mezzo a testo per richiamare
\section{chap 2}
%pacchetto natbib
inseriamo ora delle citazioni, inserendo delle citazioni va preparata però anche la bibliografia, si effettua con un begin end

\section{itemize}
%inserire item
\begin{itemize}
    \item knrgenjfn
\end{itemize}

%inserire elenco numerato
\begin{enumerate}
    \item vhvgvgvg
\end{enumerate}
\begin{thebibliography}{999}
%il numero segnato affianco è il numero massimo di citazioni che inseriremo
%tra graffe s inserisce oggetto tra quadre la carastteristica dell oggetto

\bibitem[brus et. 1993]{brus et al.}
D J Brus, J J De Gruijter
Design-based versus model-based estimates of spatial means: Theory and application in environmental soil science
Environmetrics, 4 (1993), pp. 123-152

\bibitem[Adorno 1960]{Adorno, T.W.}
Adorno, T.W. (1960) ‘The meaning of working through the past’ in Adorno, T.W. Guilt and Defense: On the legacies of National Socialism in postwar Germany (translated and edited by Olick, J.K. and Perrin, A.J., 2010). Cambridge, MA: Harvard University Press, pp. 213–27.

\end{thebibliography}
\end{document}
