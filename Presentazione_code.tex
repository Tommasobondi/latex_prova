\documentclass{beamer}
\usepackage{listings}
%NB parentesei graffe infdicano argomento di funzione
%NB parentesi quadre informazioni strutturali
\usetheme{Warsaw} %tema da utilizzare, ne puoi scegliere vari, vedi modelli su overlef
\usecolortheme{dove} %colore dominante di presentazione
%per settare effetti grafici e forme finestre ad esempio
\setbeamertemplate{blocks}[rounded][shadow=true]
%qua come in file paper, titolo ecc ecc
%inizio effettivo documento è solo dopo il titolo
\title{PRESENTAZIONE DI PROVA}
\subtitle{inserisco un sottotitolo}

\author{Tommaso Leone Bondi}
%https://it.overleaf.com/learn/latex/Inserting_Images
\institute{UniBo-Alma Mater Studiorum\\
\bigskip %con questa funzione stetto spazio immagine e testo relativo
%iserire immagine, prima va caricata su overleaf da pc
% "\\" affianco di posizione immagine la fissa li, parentsi romane aperta e poi va chiusa post funzione includegraphichs, vedi linee da institute a line 19 (})
%nb. width con rapposto setto grandezza immagine in rapporto al testo
\includegraphics[width=0.6\textwidth]{logounibo.png}
}
\begin{document}
%dobbiamo quindi inserire dtitolo dettato prima, quindi funz maketitle
\maketitle
%creare sezioni in presentazione, con indicazione di sezione in cui si è, giochino di titoli sfocati e titoli in evidenz aquando skippi slide
\section{intro}
%slide la nomino frame, facendo begin frame creo slide
%bisogna ovviamente anche settare testo
%https://it.overleaf.com/learn/latex/Font_sizes,_families,_and_styles
\begin{frame}{prima slide di intro}
\Huge provo ora con testo di grandi dimensioni
\tiny carattere più piccolo 
%NB per cambiare dimesione del testo imposto dimensione prima di testo a cui voglio assegnare la stessa
\end{frame}
\begin{frame}{seconda slide di intro}
 adsjbjadfbjkdfbjcd 
 
\end{frame}

 %apro ora nuova sezione, con relative slide
 \section{sezone centrale}
 \begin{frame}{inserimento formula} %vedi file tex precedente
 %https://www.youmath.it/come-scrivo-le-formule-matematiche.html anche per simboli
 \begin{equation}
 \sigma_{x}= \sqrt{\frac{\sum_{i=1}^{N}({x_{i}+\mu})^{2}}{N}} 
  %per inserire radici cubiche etc etc porre [] post \sqrt
 \end{equation}
 \end{frame}
 \begin{frame}{inserimento formula prova 2}
 \begin{equation}
 CV_{a.caso}=\frac {\sqrt{\frac{\sum_{i=1}^{N}({x_{i}+\mu})^{2}}{N}}}{\frac{\mu{x}}{\sqrt{\beta}\gamma}} 
  %per inserire radici cubiche etc etc porre [] post \sqrt
 \end{equation}
 \end{frame}

\end{document}
