\documentclass[a4paper,12pt]{article}
% dollaro e simbolo per inserire elemti matematici $7$
%per inserire apici $^2$
%https://texblog.org/2013/02/13/latex-documentclass-options-illustrated/
% funz \documentclass definisci tipologia di documento e carattere
% argomenti di funzioni si inseriscono tra graffe
\usepackage[utf8]{inputenc}
%con pacchetto geometry in argomenti posso settare mio documento
%https://it.overleaf.com/learn/latex/Page_size_and_margins
\usepackage{geometry}
\usepackage{linenoc} 
\usepackage{graphicx} %https://www.overleaf.com/learn/latex/Errors/File_XXX_not_found_on_input_line_XXX ->???????
%pacchetto per inserire numeri di riga
%https://it.overleaf.com/learn/latex/Inserting_Images#Introduction
\linenumbers %è funz per attivare numeri di riga
%funzione per attivare altre funzioni
% \ndvk è metodo per attivare pacchetti di latex
\title{ prova latex}
%subito sotto titlesi inseriscono autore ed affilizioni
% è simbolo per inserire commenti
\begin{document}
\begin{abstract}
    bnjvnjkv
\end{abstract}
\maketitle
%subito dopo maketitle posso inserire autori e affiliziani
%\author
\section{introduction}
%mediante funz section inserisco sezioni in mio file e poi posso scrivere normalmente
hvfbasnbfasdhadsjnjfqbhgf
\section{first chapter}
bdgjafafBDASYbadsbyj
%nb, per toglere numero da titolo utilizzo funz section con asetrisco
\section*{ second chapter}
bhjdfhyfdhukhukdfsukfdukfd
\subsection{hjdfshfhu}
hdfbhkfvbkfvbk

%con funz subsection creo sottosezioni, stesso discorso *
\subsection*{bdsbhcabhkcabkh}
njdvndvsnds (\ref{fig:Screenshot 2021-05-27 124047.png}


%titiolo è prima cosa da impostare, maketitle per farlo apparie

%per inserire figure
\begin{figure}
    \centering
    \includegraphics{Screenshot 2021-05-27 124047.png}
    \caption{screen}
    %inserire nome figura
    %far adattare testo a immagini
    \label{fig:Screenshot 2021-05-27 124047.png}
\end{figure}
\end{document}
